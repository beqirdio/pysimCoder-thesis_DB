\chap Introduction

\qquad Today we live in a heavily automated world, where almost every industry is reliant upon
multiple complex control systems. In fact, the rate at which technology is developing is very
related to our ability to control and automate the processes around us.  Considering the importance
of this field, the need for more control tools and engineers is becoming apparent. At the time,
there are already many tools which are used for designing real control systems, such as Simulink, XCos,
LTSpice, OpenModelica, CODESYS, and many more.

\quad Freely available and open-source programs are the future, therefore having control systems
tools which meet these requirements is beneficial for industry, education, and enthusiasts. One
program which complies with this idea is pysimCoder, a graphical control systems design tool which can
generate code for microcontroller units. The application is far from its paid counterparts in terms of
development. Nevertheless its open source nature allows anyone to extend the program, and is very useful
for educational purposes. PysimCoder will be the heart of my thesis work, since the main goals are related
to enhancing and extending the features of the tool. The full description of the application and its
source code can be read in chapter \ref[chap2].

\quad In order to fulfill the requirements of this thesis, my work will revolve around three main
features to be added. These will be to add vector support for the signals, to extend the support
for different hardware running on LinuxOS, as well as to test and enhance the support for NuttX RTOS
using a RISC-V architecture board. 

\quad Vectors are a topic that even children learn in school, due to their very wide usage
in various fields of mathematics. Most technical fields will use vector mathematics for one
purpose or another, including even control engineering. Therefore having the ability to work
with vectors in pysimCoder will be very useful, especially in terms of mathematical operations.
In chapter \ref[chap3] it will be explained not only why, but also how this feature can be implemented.

\quad In pysimCoder, it is already possible to generate code for real-time Linux targets. Nevertheless,
a lot of the code necessary for microcontroller hardware operations will be specific to the hardware
being used. In the case of the MZ_APO educational kit, full support will be added from scratch,
due to its previous nonexistence. The RaspberryPi already has support, however it will be extended
for 3-phase motor control demonstration. The hardware and software necessary for these implementations
will be covered in chapter \ref[chap4].

\quad NuttX is a real-time operating system (RTOS) which is is developed mainly for for small and
constrained MCU environments. Using the NuttX blocks in pysimCoder, it will be possible to test a control
system on a RISC-V based board such as the ESP32C3 development kit. In chapter \ref[chap5], I will explain
how the work done for the RasperryPi extension can be used to demonstrate motion control of a 3-Phase motor,
even using a small board with weaker capabilities.