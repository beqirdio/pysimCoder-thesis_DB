\input ctustyle3
\worktype [B/EN]
\faculty {F3}
\department {Department of Control Engineering}
\title {Open Rapid Control Prototyping, Education and Design Tools}
\author {Dion Beqiri}
\authorinfo {beqirdio@gmail.com}
\supervisor {Ing. Pavel Píša, Ph.D.}
\date {May 2022}
\abstractEN {
Control systems engineering plays a crucial role in the rapid development
of human technology, hence the demand for more user friendly and open-source
tools for systems design is significantly growing. PysimCoder is a Rapid Prototyping
Control application, which can be used to graphically design control systems
schematics for the purpose of generating real time code for different targets. Its
main advantages are  that it is freely available, open source, and it supports both
NuttX RTOS and GNU/Linux targets, all of which enhance the educational experience. 

The pysimCoder application is far from being mature, however the further
extension of the project could be a great win for education, industry, and
recreation. The main goals of this thesis will be focused on adding new features
to the project, such as support of vector signals, as well as device support for
the ESP32C3 (using NuttX RTOS) and Xilinx Zynq based MZ_APO education
kits (using GNU/Linux). Throughout this document there will be demonstrations of each
extension that was added, as well as full descriptions of the approaches used to
implement them.

}
\abstractCZ {
}

\thanks {
First and foremost, I would like to thank my supervisor for his extensive support
and mentorship during my journey.

I would also like to thank my colleagues for all the great conversations while we
were working on our theses. 

Lastly, I would like to thank all my family and friends which have believed in me
throughout my academic path...
}
\declaration {
I declare that the presented work was developed independently and that
I have listed all sources of information used within it in accordance with the
methodical instructions for observing the ethical principles in the preparation
of university theses.

In Prague 20. 5. 2022

\signature
}
\input biblio/gloss_data

\specification {%
   \vbox to0pt{\vskip-25mm\centerline{\inspic attach/assignment_bachelor_1.pdf }\vss}
   \vfil\break
   \vbox to1pt{\vskip-25mm\centerline{\inspic attach/assignment_bachelor_2.pdf }\vss}
}


\makefront

\def\example {\numberedpar C{Example}}


\input chapters/0_intro
\input chapters/1_pysimcoder
\input chapters/2_vectors
\input chapters/3_linux_gnu
\input chapters/4_nuttx
\input chapters/5_conclusion
\input chapters/6_appendix





\app Glossary
\makeglos

\bibchap
\usebib/c (iso690) biblio/sources


\bye
