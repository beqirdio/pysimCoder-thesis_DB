\chap Extending support of LinuxOS/GNU targets in pysimCoder

\qquad As mentioned in chapter 2, pysimCoder can generate code for many different real-time
operating systems. One of the main goals of this thesis is to add support for different
boards which support Linux/GNU OS. Therefore this chapter will cover all the support added
for the Xilinx Zynq based boards, and also the extension of support for the RaspberryPi boards.




\sec Zynq based board on pysimCoder

\qquad In the following section I will describe how the first implementation of a Zynq target
has been achieved. Most of the code is based on an existing Simulink project, and then adapted
for deployment in pysimCoder. 


\secc Hardware introduction

\qquad The MZ_APO educational kit is a microcontroller unit based on the Zynq-7000 SoC family,
designed by Ing. Petr Porazil at PiKRON sro. The board offers many peripherals and interfaces,
which is why it is used as an educational tool for many courses in CVUT such as Computer
Architectures, Advanced Computer Architectures, and Real-Time Systems Programming. It uses the
ARM architecture, and supports Linux/GNU as its operating system. The details of the board are
presented below (full details found at \cite[mzapo-datasheet]).

\begitems
\style 0
* Base Chip: Xilinx Zynq-7000 All Programmable SoC
* Type: Z-7010, part XC7Z010
* CPU: Dual ARM Cortex ™ -A9 MPCore ™ @ 866 MHz (NEON ™ & Single/Double Precision Floating
Point)
* 2x L1 32 kB data + 32 kB instruction, L2 512 KB
* FPGA: 28K Logic Cells (~ 430K ASIC logic gates, 35 kbit)
* Computing units in FPGAs: 100 GMACs
* FPGA memory: 240 KB
* Memory on MicroZed board: 1GB
* Operating system: GNU/Linux
* GNU LIBC (libc6) 2.19-18 + deb8u7
* Linux kernel 4.9.9-rt6-00002-ge6c7d1c
* Distribution: Debian Jessie
\enditems


\medskip \clabel[mzapo-board]{MZ_APO Educational Kit Zynq based board hardware}
\picw=6.5cm \cinspic figures/mzapo_board.png
\caption/f The hardware design of the MZ_APO kit is presented. Some main peripherals
that can be noticed easily are the LCD screen, three rotary knobs, GPIO pins, LEDs,
and the interfaces on the edges of the board. It is commonly used as educational kit
for some courses in the Czech Technical University and was designed by Ing. Petr Porazil
\cite[mzapo-datasheet]
\medskip


\quad To interface with the device, many common slots are supported such as USB-A, Gigabit
Ethernet, UART1 serial port and even a slot for microSD card storage. The peripherals
included on the board are various, as can be seen in the list below (full list at
\cite[mzapo-datasheet]).

\begitems
\style 0
* Small 16-bit parallel bus connected LCD display (480× 320, RGB 565)
* 32 LEDs for direct visualization of 32-bit word (SPI connected)
* 2× RGB LED (SPI connected, 8-bit PWM)
* 3× incremental encoder rotary knob (RGB 888, SPI connected)
* 1× 40 pin FPGA IO connector, 36 FPGA 3.3 VDC signals, jumper enables +5 VDC power,
signals match Altera DE2 kits
* 2× PMOD connectors extended by optional +5 VDC power,
each provides 8 FPGA signals shared with FPGA IO connector
* 2× parallel camera interface, one 10-bit and one 8-bit
* audio output by simple PWM modulator, on-board speaker and JACK available
* audio input to Xilinx integrated ADC, on-board microphone and JACK
\enditems

\quad Many of the peripherals above can be implemented in pysimCoder as blocks, however
for the scope of this thesis I will be working with the rotary knob "incremental encoders", 
as well as the "PMOD connectors". These two peripherals are already implemented as Simulink
blocks, as it will be presented in the next subsection. 

\quad The three incremental encoders are physical rotary knobs which can change values in
an 8-bit field (from 0 to 255). These can be useful in pysimCoder for adding variability to
a signal using some human input. The implementation will be covered in the next parts of this
section. \cite[mzapo-datasheet]

\quad The PMOD connectors are also of high interest for this thesis, as they will be the
interface between the board and some DC motor. There are two of these slots, and they will
in fact be connected to some DC motor driver, which will handle the actual output of the
voltage on the motor. \cite[mzapo-datasheet]

\secc Existing Simulink project

\qquad Since pysimCoder and Simulink are very similar in the aspect of code generation, it
is sometimes possible to translate blocks from one program to another. Thankfully, there is
already a Simulink project which can generate real-time C code for controlling the
peripherals of the MZ_APO educational kit. The source of these blocks can be found in a
public online repository \cite[mzapo-simulink], and has been developed while using the
"ert_linux" target for Simulink.

\quad "Ert_linux" is a code generation target for Simulink which is specifically tailored
for "preempt_rt" Linux distributions. The project offers an alternative to the existing
Linux code generation targets offered by MathWorks which seem to have some shortcomings.
The main advantage "ert_linux" has over the default  targets is that it uses 
"clock_nanosleep" instead of POSIX timers, which results in better real-time performance
in the "preempt_rt" Linux versions \cite[ert-linux-paper]. The source found in Github
\cite[ert-linux-repo] consists mainly of the main program template and the Makefile template,
both of which are analogous to the templates of pysimCoder used for generating real-time code
meant to run in a Linux OS environment.

\quad The "zynq-rt-control" project (also found in Github \cite[mzapo-simulink]) contains
Simulink blocks and diagrams which are used for controlling the peripherals of the MZ_APO
Zynq-based board. In order to use these resources to generate code however, it must use the
"ert_linux" target for Simulink, since it was designed using this template. Before translating
these blocks into the pysimCoder environment, it is important that the Simulink diagrams
are tested and verified on actual hardware. In this case there are only two peripheral
blocks used: knob input (incremental encoders), and DC motor (PMOD connectors). The rotary
knobs are already embedded on the board, so that can be tested out of the box. For the
testing of the other block, a DC motor and driver which connects to a PMOD connector is
required. The proper connection of such a device will be presented at the end of this
section.

\quad From the "ert_linux" repository, it will be helpful to extract which cross-compiler
must be used for generating code on the Zynq board. Then most of the C code needed for the
peripheral control will be contained in the "zynq-rt-control" repository. By combining
elements of these two projects, a similar implementation can be applied in the pysimCoder
application. Therefore the work done on the projects above will be the base of my work
on adding support for the MZ_APO educational kit.



\secc Integration with pysimCoder

\qquad In this subsection I will be explaining my approach to translating and integrating
the Simulink projects mentioned above, into the pysimCoder environment. This will be the
first implementation of a Zynq-based board, therefore a new template Makefile must be
created, as well as new blocks with all their supporting files. 

\quad To construct the template Makefile for a Linux target such as the MZ_APO kit, the
same cross-compiler specified in the "ert_linux" project should be used. The rest of the
structure of the Makefile will then be taken from the existing real-time Linux targets
supported in pysimCoder (found in source path /CodeGen/templates/rt.tmf). The compiler
specification will be switch from "gcc" to "arm-linux-gnueabihf-gcc", which ensures that
the host will be able to compile binary code compatible with the Zynq based board.

\quad Next the existing Simulink blocks must be translated into pysimCoder blocks.
Fortunately, the C code functions which the two use are very similar. Although in Simulink
there are many more functions, some of them are not related to the target manipulation,
but are rather concerned with the simulation properties. The most relevant functions are
presented below.

\begitems
\style 0
* mdlCheckParameters()
* mdlInitializeConditions()
* mdlStart()
* mdlOutputs()
* mdlUpdate()
* mdlTerminate()
\enditems

\quad On pysimCoder there are only four functions used \cite[pysimcoder-book], as presented
in the list below. The first function (INIT) is related to the initialization of the block,
and is analogous both to "mdlInitializeConditions()" and "mdlStart()" as seen above. The
periodic task functions (OUT and STUPD) are then naturally related to "mdlOutputs()" and
"mdlUpdate". Most of the core computation of the block will be implemented in these two
functions. At the end there is the termination function (END), very obviously corresponding
to "mdlTerminate()", which will handle the proper closing of the block functionality.

\begitems
\style 0
* initialization function (INIT)
* periodic task functions (OUT, STUPD)
* termination function (END)
\enditems

\quad Considering the relations shown above, the source code included in the "zynq-rt-control"
project can now be used for translation. The code from each function in Simulink, will be copied
to the corresponding function in pysimCoder. Then the main changes which will occur will be
related to the usage of parameters in the code. Both the "python_block" C structure, and the
"RCPblk" class have variables for these parameters, as can be seen in chapter 3. Each parameter
variable consists of a different data-type, therefore whatever parameter is used in the Simulink
code, will have to be converted based upon its own data-type. Parameters in pysimCoder can be
expressed as integers, real numbers, pointers, and strings. The last part of the translation will
simply be to copy all the file dependencies which are specified at the header of the C file.

\quad With the C functions all translated, a new folder will be created for the MZ_APO support,
located in the "pysimCoder/CodeGen/" directory. Inheriting the structure from the other targets,
all the .c files will be contained in a "devices" folder, the .h files in "include", and the
compiled libraries in "lib". All the file dependencies for this implementation can be found in
the "zynq-rt-control" source \cite[mzapo-simulink], and don't need any modifications. 

\quad To complete the integration of this project in pysimCoder, the supporting .xblk and .py
files must also be created for the new blocks. The properties of these block representations
will be set based on similar blocks in pysimCoder, while the parameters themselves will be
almost identical to the parameters used in Simulink. 

\quad Two new blocks will be created using this approach, based on the blocks given in the
"zynq-rt-control" \cite[mzapo-simulink] project. One will read the input given from the rotary
knobs, and the other will control a DC motor. The results of this integration will be
showcased in the following subsection. 


\secc Demonstration of results

\qquad To prove the functionality of both of the new blocks in one diagram, I have prepared a
demonstration in pysimCoder. There will be two motors used, and one of them will be following
the position of the other. The three rotary knobs embedded on the board will then be used to
tune the values of the PID controller. All the hardware necessary for this demonstration is
listed below, while the source code for running the example can be found in an online
repository \cite[mzapo-dcpid-follower].

\begitems
\style 0
* Host computer with pysimCoder installed
* 1x MZ_APO educational kit
* 2x DC motor for MZ_APO educational kit
* 1x Ethernet Cable
* 1x DC power supply (12-24)V 
* 2x Coaxial male-male cables
\enditems

\quad First the MZ_APO kit must be connected to two DC motors using the PMOD slots. Then the
three devices must be connected to the power supply in a daisy-chain pattern, using the two
coaxial cables. Last but not least, the Zynq board must be connected to the same network as the
host computer using an Ethernet cable. This will enable communication between the board and the
host. The hardware setup can be seen in the image below. 


\medskip \clabel[mzapo-dc-pid-hw]{Hardware setup for DC motor follower with PID on Zynq}
\picw=14cm \cinspic figures/DC_PID_fllwr_mzapo.jpg
\caption/f The two motors are connected to the Zynq board, while the knobs are used to tune the
PID controller. Both the motors are in the same position, which is the desired output of this
diagram.
\medskip

\quad With the hardware all set up, the diagram of the example can be opened (see Fig 4.2), and all
the parameters can be edited as needed. There are three encoder blocks, and each one of them
controls a different part of the PID controller. Although there exists a PID block in pysimCoder,
the parameters can not be variably controlled by another block. Therefore each individual element
of a PID controller has been deconstructed, and enabled for control by the knobs. One DC motor
block is fed by a constant value, which will make it the controller motor. Another DC motor is
connected on the input by the PID controller, and on the output as a negative feedback loop. 
Naturally it will be seen as the controlled motor. The TCP block at the right most part of the
diagram will then be used to propagate these signals to a TCP listener with graphing capabilities. 

\medskip \clabel[mzapo-dc-pid-diagram]{PysimCoder diagram for DC motor follower with PID on Zynq}
\picw=14cm \cinspic figures/mzapo_dcpid_demo.png
\caption/f Three knob blocks control the PID values. One DC motor is fed by some constant value,
while the other is connected in a negative feedback loop and controlled by a PID controller. The
signals are sent through TCP for recording and graphing.
\medskip

\quad Finally the code can be generated, while the sending and execution of the binary on the
Zynq board will be handled by the shell script file found in the online repository
\cite[mzapo-dcpid-follower]. Inside this file are two commands as shown below, where the "HOST"
and "TARGET" environment variables are referenced. These variables can be set through a terminal,
or can be edited in the shell script itself.

\begtt
scp DC\_PID\_follower root@\$TARGET:/tmp
ssh -t root@\$TARGET HOST=\$HOST /tmp/DC\_PID\_follower
\endtt

\quad Then the "RTScopy.py" program will be started, and the above script can be executed. Using the
TCP block in pysimCoder, some select signals will be transmitted to the host computer, and visualized
in a graph. The terminal commands for achieving this are below.


\begtt
\$ ./RTScope.py
\$ ./DC\_PID\_follower-run.sh
\endtt


\sec RaspberryPi board on pysimCoder

\qquad Raspberry Pi support on pysimCoder in fact already exists, however the translation of
an existing Simulink project \cite[rpi-rt-cntrl] can be used as a stepping stone for demonstrating
the usage of pysimCoder on an ESP32C3 board with NuttX RTOS (covered in the next chapter). Due to
this fact, the steps for implementing extended support for a RaspberryPi target will be evidently
shorter than the one for the MZ_APO kit.


\secc Hardware introduction

\qquad In the example from the "rpi-rt-control" project (available on Github \cite[rpi-rt-cntrl]),
a RaspberryPi board can control a 3-phase permanent magnet synchronous motor (PMSM) using code
generated by SimulinkCoder. The hardware necessary to test this example will be covered in the
following text.

\quad RaspberryPi is essentially a small computer or a microcontroller, which can interact with
hardware peripherals for various types of projects. It's easy operation allows people of all ages
and backgrounds to use it, making it a great asset for educational purposes \cite[rpi-mcu]. The
main technical specifications of the device used in this project can be found in the list below,
taken from the official RaspberryPi website \cite[rpi-2b], while the hardware itself can be seen
in Fig.

\begitems
\style 0
* A 900MHz quad-core ARM Cortex-A7 CPU
* 1GB RAM 
* 100 Base Ethernet
* 4 USB ports
* 40 GPIO pins
* Full HDMI port
* Combined 3.5mm audio jack and composite video
* Camera interface (CSI)
* Display interface (DSI)
* Micro SD card slot
* VideoCore IV 3D graphics core
\enditems

\medskip \clabel[rpi-board]{RaspberryPi Model 2B v1.1}
\picw=11cm \cinspic figures/rpi-2b-v1.png
\caption/f The physical appearance of the specific RaspberryPi used in the PMSM control
demonstration for Simulink/pysimCoder. Produced in 2014, it is a Model 2B (v1.1).
\medskip

\quad Unfortunately a 3-phase PMSM cannot simply be plugged into this board and used out of
the box. For this reason, additional hardware will be required, such as a 3-phase motor driver,
and a FPGA expansion unit. Luckily, these devices have already been produced and provided by a
Bachelor's thesis work done by Martin Prudek \cite[fpga-thesis]. Therefore this hardware can in
fact simply be connected and used, especially for the purpose of motor control (see Fig 4.5).

\medskip \clabel[rpi-pmsm-setup]{Hardware setup for PMSM control w/ RPi board}
\picw=11cm \cinspic figures/rpi_pmsm.png
\caption/f Hardware setup for motion control of 3-phase motor (PMSM) using RaspberryPi board
and Simulink. The motor is connected to the motor driver, the FPGA board is fitted on top
of the driver, and the RaspberryPi is connected to the expansion board with a HAT cable module.
Note that the board in this picture is one model older than the one used in my implementation.
\medskip


\quad The motor driver board's main task is to connect all of the I/O cables of the motor to the
expansion unit. This includes connections for the IRC, the Hall effect sensors, and the 3-phase
windings of the motor. It will also serve as the power source of the motor and the boards
attached to it, once it is connected to a DC power supply. Other components include an ADC,
half-bridges, and current measurement signals.

\quad Even with the current amount of hardware, the microcontroller unit still cannot be attached
and used. This is due to the fact that the specific RaspberryPi board used for this example does
not have any IRC input, and only contains one PWM output, making it unfit for motion control
situations \cite[ert-linux-paper]. Fortunately, the FPGA expansion unit was specifically
designed to act as a middle-man between the motor driver, and the pins of the RaspberryPi board.


\quad Field programmable gate arrays (FPGAs) are semi-conductor devices which can (re)program
configurable logic blocks in order to achieve a certain logical system design. The specific FPGA
design for this example can be used in a few different modes. For example, one mode is used for
motion control of a DC motor, where it converts the IRC signals, and switches PWM to be usable
in both voltage directions. Then there is the mode relevant to this project, which will enable
the motion control of a 3-phase motor. The fully described FPGA design can be found in Martin
Prudek's Bachelor thesis \cite[fpga-thesis].

\quad To communicate with the FPGA expansion unit, the RaspberryPi board will use its SPI pins.
Then the source of the FPGA will be able to send and receive signals from the MCU to the motor
driver successfully. In addition to this, the expansion unit also holds a DC voltage converter
which will be able to give a stable 5 Volts that the MCU needs for operation, while the 3.3V
pins of the MCU will route back to the expansion kit for operation of the FPGA circuitry. More
details about this hardware setup and the PMSM Simulink example can be found online
\cite[fpga-realtime].


\secc Existing Simulink project and translation to pysimCoder

\qquad As mentioned previously, I will use an existing project made for Simulink, in order to
make a new block for PMSM control in pysimCoder. The files of interest for the following
implementation can be found in the "rpi-rt-control" online repository \cite[rpi-rt-cntrl]. Only
the files related to the PMSM motion control will be used, which include the block C file and
its dependencies, as well as the Simulink diagram used for the control system.

\quad For the conversion of the Simulink block to a pysimCoder block, the same approach from the
previous section will be utilized.  The blocks of code will be extracted from each Simulink
function and copied onto the corresponding pysimCoder function. Then all the parameters must be
adapted to the python_block structure, and the supporting header files specified on the top of
the C file. All the dependencies mentioned must also be copied onto the "devices" and "include"
folders of the RaspberryPi target, in order to enable the proper compilation of the block's C
code.

\quad In this particular Simulink block, there are also vector inputs and outputs. To avoid
possible issues, I have simply  split all these signals into individual inputs and outputs,
rather than using the experimental vector feature in pysimCoder. This decision will ensure
that the cause of some error message will more likely be related to the new code which was
added.

\quad A RaspberryPi target for cross-compilation is already supported in pysimCoder, and in
fact there is already a whole library of blocks to generate code for these boards. Therefore,
all the block related files (.xblk, .py, .c) which were adapted for the new PMSM block can
simply be added to their corresponding folders. The C file in particular should also be
compiled in its folder using the "make" command in a terminal, so that the code generation
tool will be able to use the new block.  In other words, there is no need to create any folders,
or to construct any Makefile templates.

\quad There is also a Simulink diagram which can be used to test the functionality
of the block on real hardware. The same principles of this diagram will be emulated in the
pysimCoder environment, as it has all the blocks necessary to do so. The implementation will
be covered in the following subsection. 


\secc Demonstration of results

\qquad To test whether the adaptation of the block has succeeded, a real-life example should be
tested. Therefore all the hardware must initially be set up (as seen in Fig. 4.5), and a diagram
in pysimCoder should be constructed to demonstrate the motion control of a 3-phase motor. 

\quad Fortunately, most of the blocks used in the Simulink diagram, can also be found
in the pysimCoder library. Nevertheless the diagram will be simplified and cut down so that only
the necessary functions will be used. Some parts of the diagram in Simulink have a lot to do with
graphing, however for the purpose of this demonstration only the signals of higher interest will
be visualized.


\medskip \clabel[rpi-pmsm-dgm]{PysimCoder diagram for motion control of 3-Phase Motor}
\picw=15cm \cinspic figures/rpi_pmsm_dgm.png
\caption/f PysimCoder diagram for control of 3-phase PMSM using RaspberryPi target. Design has
similarities to the original Simulink diagram, however it has been reduced only to the necessary
control signals and blocks.
\medskip

\quad The new RaspberryPi block has 6 inputs, and 7 outputs. The first three inputs will be the
PWM signals being fed to the motor, while the last three inputs are "enable" signals (which allow
usage of a certain PWM signal). Then on the first three outputs are measurements of the current
output, one for each phase. The next three are outputs related to the IRC, and the last output
will be for the HAL sensor measurements. 

\quad The first three inputs of the PMSM block will be supplied by an "Inverse Clarke" math
operation block, which is previously fed by an "Inverse Park" block. Into the latter, there will
be one constant signal (zero for this case), one PID controlled signal, as well as a feedback signal
from the processed HAL measurement. The pulse wave generator seen on the far left, will be the input
for this control system, and it is subtracted with the IRC output before being fed to the PID
controller. At the very right there is also a TCP block, which will transfer select signal data
through a TCP/IP connection with the host computer. This connection can then be utilized by the
RTScope.py program in order to graph all the incoming data points (same as done in last section).

\quad With the diagram all set up, the C code can finally be generated. The binary will simply
be transferred and executed on the RaspberryPi, considering all the hardware is connected properly.
Although it is possible to interact with the board using the serial terminal, it is recommended
to use an SSL connection instead. Therefore the RaspberryPi and the host must be connected to the
same network (same as for the MZ_APO demonstration). On the host, the RTScope.py program must be
started to enable the receiving and visualization of the incoming data. A preview of this example
can be seen in Fig 4.7. 

\medskip \clabel[rpi-pmsm-graph]{Graph of PMSM motion control example with RaspberryPi}
\picw=15cm \cinspic figures/rpi_pmsm_graph.png
\caption/f Graph showing select signals from the PMSM motion control example using the RaspberryPi.
The vertical axis will be dimensionless, since the signals graphed do not have the same units or
range of values for that matter. The horizontal axis represents time in seconds. This graph is a
screenshot taken from the RTScopy.py program, and is only for testing purposes. 
\medskip

\quad The graph above is the result of the control system applied in the RaspberryPi board for
motion control of a PMSM. In red is the input signal being fed to the system, while in green is the
IRC position. It is clearly seen that the two signals are in sync when there is no disturbance to
the system. However, as seen from the blue signal, once a disturbance is applied the controller will
try its best to compensate. The IRC output seems to be slightly shifted to the right, however it
still follows the same rate of change as given from the input signal. Once the disturbance is
removed, the IRC immediately synchronizes with the input again. This is in fact the desired system
response for this setup, therefore it has been proved that the new PMSM block has been successfully
integrated into pysimCoder. 
