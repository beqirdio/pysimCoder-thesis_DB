\app Source Code

\qquad Here will be listed all the features added to the project, and a corresponding
reference to the online contribution to the project on Github (if applicable).



\sec Vectors

\qquad The source code of the vector features are still in development, therefore they
are located in my fork of the original "pysimCoder" project. The two branches are listed:

\begitems
\style O

* Vector support with static dimension setting 
\urlnote{https://github.com/beqirdio/pysimCoder/tree/vectors}

* Automatic dimension-setting algorithm for vector support 
\urlnote{https://github.com/beqirdio/pysimCoder/tree/vectors_v2_0}

\enditems



\sec Linux Targets

\qquad For the two boards which were running GNU/Linux, the main contributions
to the "pysimCoder" and "pysimCoder-examples" repositories are listed:

\begitems
\style O

* New blocks (DC motor, encoder) and targets  for Xilinx Zynq based MZ_APO educational kit
\urlnote{https://github.com/robertobucher/pysimCoder/pull/31}

* Demonstration of PID controlled motor follower for MZ_APO educational kit in diagram
\urlnote{https://github.com/robertobucher/pysimCoder-examples/pull/2}

* New block (PMSM) for motion control using RaspberryPi target
\urlnote{https://github.com/robertobucher/pysimCoder/pull/49}

* Demonstration of PMSM RaspberryPi block in diagram (using TCP protocol)
\urlnote{https://github.com/robertobucher/pysimCoder-examples/tree/main/Linux-mzapo/DCmotor}

\enditems



\sec NuttX Targets

\qquad For the demonstration realized with the ESP32C3 board, a new block and diagram has been
created for NuttX targets:

\begitems
\style O

* New block (PMSM) for 3-phase motion control using NuttX
\urlnote{https://github.com/robertobucher/pysimCoder/pull/49}

* Demonstration of PMSM NuttX block in a diagram (using CAN protocol)
\urlnote{https://github.com/beqirdio/pysimCoder-examples/tree/main/NuttX/3_phase_PMSM_control}

\enditems